\documentclass[12pt]{beamer}

\usetheme{Air}
\usepackage{thumbpdf}
\usepackage{wasysym}
\usepackage{ucs}
\usepackage[utf8]{inputenc}
\usepackage[french]{babel}
\usepackage{pgf,pgfarrows,pgfnodes,pgfautomata,pgfheaps,pgfshade}
\usepackage{verbatim}

\pdfinfo
{
  /Title       (Détection de citations implicites)
  /Author      (Grégoire Jadi)
  /Author      (Joseph Lark)
}


\title{Détection de citations implicites}
\subtitle{Article de V. Qazvinian \& D. Radev (ACL 2010)}
\author{Grégoire Jadi \& Joseph Lark}
%\date{September 6th 2006}

\begin{document}

\frame{\titlepage}

\section*{}
\begin{frame}
  \frametitle{Plan}
  \tableofcontents[section=1,hidesubsections]
\end{frame}

\AtBeginSection[]
{
  \frame<handout:0>
  {
    \frametitle{Plan}
    \tableofcontents[currentsection,hideallsubsections]
  }
}

\AtBeginSubsection[]
{
  \frame<handout:0>
  {
    \frametitle{Plan}
    \tableofcontents[sectionstyle=show/hide,subsectionstyle=show/shaded/hide]
  }
}

\newcommand<>{\highlighton}[1]{%
  \alt#2{\structure{#1}}{{#1}}
}

\newcommand{\icon}[1]{\pgfimage[height=1em]{#1}}



%%%%%%%%%%%%%%%%%%%%%%%%%%%%%%%%%%%%%%%%%
%%%%%%%%%% Content starts here %%%%%%%%%%
%%%%%%%%%%%%%%%%%%%%%%%%%%%%%%%%%%%%%%%%%



\section{Contexte}

\begin{frame}
  \frametitle{Contexte}
  \begin{block}{Problématique}
  \begin{itemize}
    \item Lier un article aux travaux précédents
    \item Identifier le contexte d'un article
  \end{itemize}
  \end{block}

  \begin{block}{Objectif de l'article}
  \begin{itemize}
    \item Détecter les citations implicites dans un article
    \item Générer un résumé d'article
  \end{itemize}
  \end{block}
\end{frame}

\section{Travaux précédents}
\begin{frame}
  \frametitle{Travaux précédents}
  \begin{itemize}
    \item Réseaux de citations et collaborations [2001,2006]
    \item Importance des citations dans l'identification de contexte [2008]
    \item Catégories de citations pour la génération de résumé [2004,2009]
  \end{itemize}
    
\end{frame}

\section{Méthode proposée}


\begin{frame}
  \frametitle{Cadre de travail}
  \begin{block}{Données}
  \begin{itemize}
    \item 10 articles récents de l'ANN [2003-2008]
    \item 203 couples article-citation
    \item environ 1500 phrases
  \end{itemize}
  \end{block}
\end{frame}

\begin{frame}
  \frametitle{Vecteurs de contexte}
  Pour un article donné A, un article donné B,
  Vect(A,B) tel que pour chaque phrase P de l'article B, P-ième valeur de Vect(A,B) :
  \begin{itemize}
    \item 0 si aucune citation
    \item 1 si citation implicite
    \item C si citation explicite
  \end{itemize}
\end{frame}

\begin{frame}
  \frametitle{Fiabilité de l'annotation}
  Un annotateur expert mais exterieur est employé
  La fiabilité est mesurée par le coefficient kappa (rappel?)
\end{frame}

\begin{frame}
  \frametitle{Une première analyse}
  \begin{itemize}
    \item Distribution article / nombre de citations
    \item Positions des citations implicites / explicites
  \end{itemize}
\end{frame}

\begin{frame}
  \frametitle{Méthode}
  Les étapes
  \begin{itemize}
    \item Modélisation du réseau de phrases par un MRF
    \item Belief Propagation
    \item Score et décision
  \end{itemize}
\end{frame}

\begin{frame}
  \frametitle{Markov Random Fields}
  \begin{block}{But}
  \begin{itemize}
   \item Inférence
   \item Classification
  \end{itemize}
  \end{block}
  
  \begin{block}{Structure}
  \begin{itemize}
   \item Graphe non orienté
   \item Noeuds observés et noeuds cachés
  \end{itemize}
  \end{block}
  
  \begin{block}{Fonctionnement}
  L'état de chaque noeud caché dépend de la valeur du noeud observé correspondant,
  et de l'état des noeuds cachés voisins.
  \end{block}
\end{frame}

\begin{frame}
  \frametitle{Modélisation}
  \begin{itemize}
   \item Chaque phrase est modélisée par un noeud dans le graphe, qui est affecté de deux scores (contexte et non)
   \item Le nombre de dépendances est variable (c'est un paramètre testé).
   \item On note BPi le MRF où chaque noeud est lié à 2i voisins (i phrases précédentes, i phrases suivantes)
  \end{itemize}
\end{frame}

\begin{frame}
  \frametitle{Belief Propagation}
  Les scores de chaque noeud du graphe dépendent de ceux des noeuds voisins par message passing

  ...
\end{frame}

\begin{frame}
  \frametitle{Score et décision}
  (equation 1 de l'article)
  ...
\end{frame}

\section{Experiences et résultats}
\begin{frame}
  \frametitle{Baselines}
  \begin{block}{Baseline "RI"}
  %\begin{itemize}
    Mesure de similarité entre les phrases contenant des citations
  %\end{itemize}
  \end{block}
  \begin{block}{Baseline "MRF"}
  %\begin{itemize}
    Cherche des motifs de discours dans le voisinnage des citations 
  %\end{itemize}
  \end{block}
\end{frame}

\begin{frame}
  \frametitle{Résultats}
  tableau de résultats
\end{frame}

\section{Conclusion}
\begin{frame}
  \frametitle{Conclusion}

  \begin{itemize}
    \item amélioration de ...
    \item travaux à venir
  \end{itemize}
\end{frame}

\end{document}

